\section{Lecture 19}
\subsection{Some MORE Properties}
\begin{enumerate}
    \item[5.5.] Product, again.
    \begin{align}
        \widehat{f \cdot g}(\xi) = \dfrac{1}{\sqrt{2\pi}} \cdot (\hat{f} * \hat{g})(\xi)
    \end{align}
    \begin{proof}
        Taking the FT of the left side (use Property 10) we get
        \begin{align}
            (f \cdot g)(-x)
        \end{align}
        Taking the FT of the right side, via Property 5, we get
        \begin{align}
            \dfrac{1}{\sqrt{2\pi}} \cdot \sqrt{2\pi} \cdot \hat{\hat{f}} \cdot \hat{\hat{g}}
        \end{align}
        which is equivalent to
        \begin{align}
            f(-x) \cdot g(-x)
        \end{align}
        which shows that both sides are the same.
    \end{proof}
    \item[7.] If $f$ vanishes outside $0, 1$, then
    \begin{align}
        a_n(f) = \sqrt{2\pi} \cdot \hat{f}(2\pi n)
    \end{align}
    \begin{proof}
        \begin{align}
            a_n(f) &= \int_0^1 f(t) e^{-2\pi in t} \dd{t}\\
            &= \int_{-\infty}^\infty f(t) e^{-2\pi in t} \dd{t}\\
            &= \int_{-\infty}^\infty f(t) e^{-i (2\pi n) t} \dd{t}\\
            &= \sqrt{2\pi} \cdot \hat{f}(2\pi n)
        \end{align}
    \end{proof}
    \item[8.] Plancharel's Identity.\footnote{This is a mess but it does make sense.}
    \begin{align}
        \norm{\hat{f}} = \norm{f}
    \end{align}
    This can be proven for the special case, all functions that vanish outside some finite interval $[a, b] \subset \mathbb{R}$.
    \begin{proof}
        Using translation,
        \begin{align}
            \hat{f}(\xi + a) = \hat{f}(\xi) \cdot e^{i\xi a}
        \end{align}
        Define
        \begin{align}
            g(x) := f(x + a)
        \end{align}
        Then, 
        \begin{align}
            \norm{g} = \int \abs{f(x+a)}^2 \dd{x} = \int \abs{f(y)}^2 \dd{y}
        \end{align}
        And,
        \begin{align}
            \norm{\hat{g}} &= \int \abs{\hat{f}(\xi) \cdot e^{i\xi a}}^2 \dd{\xi}\\
            &= \int \abs{\hat{f}(\xi)}^2 \dd{\xi}\\
            &= \norm{\hat{f}}^2
        \end{align}
        So
        \begin{align}
            \norm{\hat{g}} = \norm{g} \implies \norm{\hat{f}} = \norm{f}
        \end{align}
        So, we want to prove this true for all functions $g$, which vanish on the interval $[0, b-a]$. That is, this restates the theorem as needing to prove
        \begin{align}
            \norm{\hat{g}} = \norm{g}
        \end{align}
        for all functions that vanish outside
        \begin{align}
            [0, C] \subset \mathbb{R}
        \end{align}
        Using dilation, define
        \begin{align}
            h(x) := g(C \cdot x)
        \end{align}
        Then,
        \begin{align}
            \norm{h}^2 &= \int \abs{g(C\cdot x)}^2 \dd{x}\\
            &= \dfrac{1}{C} \cdot \int \abs{g(y)}^2 \dd{y}\\
            &= \dfrac{1}{C} \cdot \norm{g}^2
        \end{align}
        and
        \begin{align}
            \norm{\hat{h}}^2 &= \int \abs{\dfrac{1}{C} \hat{g}(\frac{\xi}{C})} \dd{\xi}\\
            &= \int \abs{\frac{1}{C} \cdot \hat{g}(\eta)}^2 \cdot \abs{C} \dd{\eta}\\
            &= \frac{1}{C} \norm{\hat{g}}^2
        \end{align}
        Due to $h$ being defined as a dilation, $h$ vanishes outside $[0,1]$. So, it suffices to prove this Lemma for functions that vanish outside $[0,1]$, which is already proven in (19.2-6). That is, suppose $\hat{f}$ vanishes outside $[0,1]$. By Property 7,
        \begin{align}
            a_n(f) = \sqrt{2\pi} \cdot \hat{f}(2\pi n)
        \end{align}
        Now, we can use modulation. Define
        \begin{align}
            g_t(x) := f(x) \cdot e^{2\pi it x}
        \end{align}
        Then,
        \begin{align}
            \norm{g_t}^2 = \int \abs{f(x)}^2 \dd{x} = \norm{f}^2
        \end{align}
        and,
        \begin{align}
            a_n(g_t) = \sqrt{2\pi} \cdot \hat{g_t}(2\pi n) = \sqrt{2\pi} \cdot \hat{f}(2\pi n + t)
        \end{align}
        Then,
        \begin{align}
            2\pi \norm{f}^2 = \int_0^{2\pi} \norm{f}^2 \dd{t} &= \int_0^{2\pi} \norm{g_t}^2 \dd{t} = \int_0^{2\pi} \sum \abs{a_n(g_t)}^2 \dd{t}\\
            &= \int_0^{2\pi} 2\pi \sum \abs{\hat{f}(2\pi n + t)}^2 \dd{t}\\
            &= 2\pi \sum \int_0^{2\pi} \abs{\hat{f}(2\pi n + t)}^2 \dd{t}
        \end{align}
        Use the substitution $s = 2\pi n + t$ and we get
        \begin{align}
            2\pi \sum \int_{2\pi n}^{2\pi (n + 1)} \abs{\hat{f}(s)}^2 \dd{s}
        \end{align}
        or
        \begin{align}
            2\pi \int \abs{\hat{f}(s)}^2 \dd{s} = 2\pi \norm{\hat{f}}^2
        \end{align}
        so, by transitivity, (19.31) equals (19.35), so
        \begin{align}
            \norm{f}^2 = \norm{\hat{f}}^2
        \end{align}
    \end{proof}
\end{enumerate}