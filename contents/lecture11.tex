\section{Lecture 11}
``We're closing the parentheses; last class was the parentheses.''

\subsection{Convergence of Fourier Series}
We showed
\begin{lemma}
    If $f: \mathbb{R} \to \mathbb{R}$ is continuous, differentiable, and periodic with period $1$, then
    \begin{align}
        (S_Nf)(x) \to_{N \to \infty} f(x) && \text{pointwise}
    \end{align}
\end{lemma}
A similar proof of this theorem leads to a stronger statement
\begin{lemma}
    If $f: \mathbb{R} \to \mathbb{R}$ continuous, continuously differentiable except at finitely many points $\in [0,1]$, and periodic with period $1$, then,
    \begin{align}
        (S_Nf)(x) \to_{N\to\infty} f(x) && \text{uniformly}
    \end{align}
\end{lemma}
We can extend this to norm-wise convergence
\begin{lemma}
    For every $h \in C[0,1]$, $\norm{S_Nh} \le \norm{h}$
    \begin{proof}
        Bessel's Inequality says this, but with squares.
        \begin{align}
            \norm{S_Nh}^2 \le \norm{h}^2
        \end{align}
    \end{proof}
\end{lemma}
The next Lemma is less trivial
\begin{lemma}
    If $f \in C[0,1]$ then there exists a sequence $g_N$ such that
    \begin{enumerate}
        \item Each $g_N$ is continuous, continuously differentiable except at finitely many points, and periodic with period 1
        \item $g_N \to f$ in norm
    \end{enumerate}
\end{lemma}
Then, recall
\begin{theorem}
    $\{ e^{2\pi int} \mid n \in \mathbb{Z} \}$ is a complete orthonormal set in $C[0,1]$.
\end{theorem}
\begin{proof}
    Need only to show completeness, i.e., that if $f \in C[0,1]$, then $S_Nf \to f$ in norm. By Lemmma 18, there is a sequence $g_N$ that $g_N \to f$ in norm.
    \begin{align}
        \norm{S_Nf - f} &= \norm{S_Nf - S_Ng_k + S_Ng_k - g_k + g_k - f}\\
        &= \norm{(S_Nf - S_Ng_k) + (S_Ng_k - g_k) + (g_k - f)}
    \end{align}
    By the Triangle Inequality,
    \begin{align}
        (11.5) &\le \norm{S_Nf - S_Ng_k} + \norm{S_Ng_k - g_k} + \norm{g_k - f}
    \end{align}
    Given $\varepsilon$, there is $k$ such that
    \begin{align}
        \norm{f - g_k} < \varepsilon
    \end{align}
    Take this $k$
    \begin{align}
        \norm{S_Nf - S_Ng_k} &= \norm{S_N(f-g_k)}\\
        &\le \norm{f-g_k}\\
        &< \varepsilon
    \end{align}
    Next, by Lemma 11.2,
    \begin{align}
        S_Ng_k \to g_k && \text{ uniformly}
    \end{align}
    i.e., there is $N_0$ such that $\forall N > N_0$, for all $x$,
    \begin{align}
        \abs{S_Ng_k - g_k} \le \varepsilon
    \end{align}
    Then,
    \begin{align}
        \norm{S_Ng_k - g_k}^2 = \int_0^1 \abs{S_Ng_k(x) - g_k(x)}^2 \dd{x} \stackrel{(11.12)}{<} \varepsilon^2
    \end{align}
    So
    \begin{align}
        \norm{S_Ng_k - g_k} \le \varepsilon
    \end{align}
    So
    \begin{align}
        \norm{(S_Nf - S_Ng_k) + (S_Ng_k - g_k) + (g_k - f)} \le 3\varepsilon
    \end{align}
    Then, use $\varepsilon := \varepsilon/3$ and
    \begin{align}
        \norm{(S_Nf - S_Ng_k) + (S_Ng_k - g_k) + (g_k - f)} \le \varepsilon
    \end{align}
    Proving that
    \begin{align}
        S_Nf \to f
    \end{align}
\end{proof}
Lemma 11.4, however, is still unproven, and can be proven using pictures.\footnote{????????}