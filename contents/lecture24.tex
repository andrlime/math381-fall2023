\section{Lecture 24}
\subsection{Some more on the $n$-D Fourier Transform}
Recall
\begin{align}
    \widehat{f}(\xi, \eta) = \dfrac{1}{2\pi} \iint_{\mathbb{R}^2} f(x, y) e^{-i(\xi x + \eta y)} \dd{x}\dd{y}
\end{align}
Note that
\begin{align}
    \xi x + \eta y = \mqty[\xi\\\eta] \cdot \mqty[x\\y]
\end{align}
So, a more general definition is
\begin{definition}
    A Fourier transform of some function can be written as
    \begin{align}
        \hat{f}(\vb{v}) = \dfrac{1}{(2\pi)^{n/2}} \int_{\mathbb{R}^n} f(\vb{x}) e^{-i(\vb{x} \cdot \vb{v})} \dd{\vb{x}}
    \end{align}
\end{definition}

\begin{definition}
    (Inverse Fourier Transform)
    \begin{align}
        f(\vb{x}) = \dfrac{1}{(2\pi)^{n/2}} \int_{\mathbb{R}^n} \hat{f}(\vb{v}) \cdot e^{i(\vb{x} \cdot \vb{v})} \dd{\vb{v}}
    \end{align}
\end{definition}
\begin{proposition}
    If $f(x,y) = g(x)h(y)$, then
    \begin{align}
        \hat{f}(\xi, \eta) &= \dfrac{1}{2\pi} \iint f(x,y) e^{-i(\xi x + \eta y)} \dd{x}\dd{y}\\
        &= \dfrac{1}{2\pi} \iint g(x)e^{ix\xi} \cdot h(y) e^{iy\eta} \dd{x}\dd{y}\\
        &= \dfrac{1}{2\pi} \int g(x)e^{ix\xi} \dd{x} \cdot \int h(y) e^{iy\eta} \dd{y}\\
        &= \hat{g}(\xi) \cdot \hat{h}(\eta)
    \end{align}
\end{proposition}
So, we can apply this to a Gaussian distribution; we get
\begin{align}
    \mathcal{F}\left[ \exp(-\dfrac{x^2+y^2}{2}) \right] = \exp[-\dfrac{\xi^2+\eta^2}{2}]
\end{align}

\subsection{Pulses}
Let
\begin{align}
    1_{[-1/2, 1/2] \times [-1/2, 1/2]}(x,y) = \begin{cases}
        1 & (x,y) \in [-1/2, 1/2] \times [-1/2, 1/2]\\
        0 & \text{else}
    \end{cases}
\end{align}
This is actually just the multiplication of one-dimensional square pulses, which we know the FT of. So, the FT of this function equals the product, which is
\begin{align}
    \text{sinc}(x) \cdot \text{sinc}(y)
\end{align}

\subsection{Radial Invariance}
If $f(x,y)$ is radial, then $\hat{f}(\xi, \eta)$ is also radial. To compute it, we might as well just compute it at $(r, 0)$, as all of the other directions are just rotations of this. Note that since $f(x, y)$ is radial, it is some function
\begin{align}
    \phi(\sqrt{x^2 + y^2})
\end{align}
Using the 2D FT,
\begin{align}
    2\pi \hat{f}(r, 0) = \iint f(x, y) e^{-irx} \dd{x}\dd{y}
\end{align}
Then, change to polar coordinates
\begin{align}
    \iint \phi(r) e^{-ir^2\cos(\theta)} r\dd{r}\dd{\theta}
\end{align}
One of these integrals, nobody knows the solution to. So, give it a name
\begin{align}
    \int_0^{2\pi} e^{-ix\cos(\theta)} \dd{\theta} = 0\text{-th Bessel Function $J_0(x)$}
\end{align}
Now, this is the end of this document. There was a small amount of content covered after this, but it does not fit well with the themes of this course, and I also did not take notes for them, as they felt very tangential.