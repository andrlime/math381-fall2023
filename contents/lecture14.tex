\section{Lecture 14}
\subsection{More on the Heat Equation / Diffusion Equation}
The equation
\begin{align}
    \pdv{u}{t} = \kappa \pdv[2]{u}{x}
\end{align}
Suppose we want to solve this on the interval
\begin{align}
    [0, L]
\end{align}
as opposed to
\begin{align}
    [0, 1]
\end{align}
Generally, on $[0, 1]$, the set of functions
\begin{align}
    \{ e^{2\pi in x} \}
\end{align}
is used as an orthonormal basis. On $[0, L]$ we instead use
\begin{align}
    \{ e^{\frac{2\pi in x}{L}} \}
\end{align}
as the orthogonal basis.
\begin{proof}[Solution]
    Expand $u(t, x)$ using the new basis in (14.5), i.e.
    \begin{align}
        u(t, x) = \sum U_n(t) \cdot e^{\frac{2\pi in x}{L}}
    \end{align}
    Now,
    \begin{align}
        \pdv{u}{t} &= \sum U_n'(t) \cdot e^{\frac{2\pi in x}{L}}\\
        &= \kappa \pdv[2]{u}{x}\\
        &= \sum U_n(t) \cdot \kappa\left( \frac{-4\pi^2 n^2}{L^2} \right) \cdot e^{\frac{2\pi in x}{L}}
    \end{align}
    So,
    \begin{align}
        U_n'(t) &= U_n(t) \cdot \kappa\left( \frac{-4\pi^2 n^2}{L^2} \right)\\
        &= \frac{-4\pi^2 n^2 \kappa}{L^2} U_n(t)
    \end{align}
    The solution to this is
    \begin{align}
        c_n e^{\frac{-4\pi^2n^2\kappa}{L^2}t} \cdot e^{\frac{2\pi in x}{L}}
    \end{align}
    So, the general solution becomes
    \begin{align}
        \boxed{\sum_n c_n e^{\dfrac{-4\pi^2n^2\kappa}{L^2}t} \cdot e^{\dfrac{2\pi in x}{L}}}
    \end{align}
    To find $c_n$, do
    \begin{align}
        \left\langle u(0, x), e^{\frac{2\pi in x}{L}} \right\rangle &= \int_0^L \left[ h(x) \cdot e^{\frac{2\pi in x}{L}} \right] \dd{x}\\
        &= \left\langle \sum_k c_k e^{\frac{2\pi ik x}{L}}, e^{\frac{2\pi in x}{L}} \right\rangle\\
        &= c_n \int_0^L 1 \dd{x} & \text{all vanish except $k = n$}\\
        &= c_nL
    \end{align}
    so
    \begin{align}
        \boxed{c_nL = \int_0^L \left[ h(x) \cdot e^{\frac{2\pi in x}{L}} \right] \dd{x}}
    \end{align}
    \begin{lemma}
        \begin{align}
            \lim_{t \to \infty} u(t, x) &= c_n\\
            &= \frac{1}{L} \int_0^L h(x) \dd{x}
        \end{align}
        i.e. as time goes to infinity, the temperature of all points on some one-dimensional rod converge to the average temperature of the rod.
        \begin{proof}
            Pretty easy based on the above equations. Take $t \to \infty$ in (14.13).
        \end{proof}
    \end{lemma}
\end{proof}

\subsection{Continuity}
\begin{proposition}
    If $f: [0, 1] \to \mathbb{C}$ is continuous, then
    \begin{enumerate}
        \item The Fourier coefficients tend to zero
        \begin{align}
            a_n(f) \to 0
        \end{align}
    \end{enumerate}
\end{proposition}
\begin{proof}
    Implied by Lemma 9.5
\end{proof}
\begin{proposition}
    If
    \begin{align}
        \sum \abs{a_n} < \infty && \text{absolutely converges}
    \end{align}
    then the series
    \begin{align}
        \abs{a_n \cdot e^{2\pi in x}}
    \end{align}
    converges to a continuous function.
\end{proposition}
\begin{proof}
    \begin{enumerate}
        \item The sum in (14.23) absolutely converges
        \begin{align}
            \sum \abs{a_n \cdot e^{2\pi in x}} = \sum \abs{a_n} <  \infty
        \end{align}
        (note the cancellation because $\abs{e^{2\pi in x}} = 1$) so thus also converges.
        \item Let $f(x) = \sum a_n e^{2\pi in x}$. We claim that
        \begin{align}
            \sum a_n e^{2\pi in x} \to f(x)
        \end{align}
        uniformly.
        \begin{proof}
            Let $\varepsilon > 0$ be given. Since
            \begin{align}
                \sum \abs{a_n} < \infty
            \end{align}
            there is some $N$ such that
            \begin{align}
                \sum_{\abs{n} > N} \abs{a_n} < \varepsilon
            \end{align}
            So, for every $x$,
            \begin{align}
                \abs{f(x) - \sum_{n = -N}^N a_n e^{2\pi in x}} < \varepsilon
            \end{align}
            Removing these elements yields
            \begin{align}
                \abs{\sum_{\abs{n}>N} a_n e^{2\pi in x}} &\le \abs{\sum_{\abs{n}>N} a_n e^{2\pi in x}}\\
                &\le \sum_{\abs{n}>N} \abs{a_n} \cdot \abs{{e^{2\pi in x}}}\\
                &\le \sum_{\abs{n}>N} \abs{a_n}\\
                &\le \varepsilon
            \end{align}
        \end{proof}
        \item We now claim $f(x)$ is continuous, so we need to show that $\forall x_0 \in [0, 1]$, and $\varepsilon > 0$ if $h$ is small enough,
        \begin{align}
            \abs{f(x_0 + h) - f(x_0)} < \varepsilon
        \end{align}
        \begin{proof}
            \begin{align}
                \abs{f(x_0 + h) - f(x_0)} &= \abs{\sum a_n \left[ e^{2\pi in (x_0 + h)} - e^{2\pi in x_0} \right]}\\
                &= \abs{\sum a_n e^{2\pi in x_0} \left[ e^{2\pi in h} - 1 \right]}\\
                &\le \sum \abs{a_n e^{2\pi in x_0}} \cdot \abs{ e^{2\pi in h} - 1}
            \end{align}
            This equals
            \begin{align}
                (14.36) &= \sum_{\abs{n} \le N} \abs{a_n e^{2\pi in x_0}} \abs{ e^{2\pi in h} - 1} + \sum_{\abs{n} > N} \abs{a_n e^{2\pi in x_0}} \abs{ e^{2\pi in h} - 1}\\
                &\le \sum_{\abs{n} \le N} \abs{a_n e^{2\pi in x_0}} \abs{ e^{2\pi in h} - 1} + 2\sum_{\abs{n} > N} \abs{a_n}
            \end{align}
            Claim: the inequality
            \begin{align}
                (14.38) \le 14MN^2\abs{h} + 2\varepsilon
            \end{align}
            is true.
            \begin{proof}[Reasoning]
                After plugging in (14.32), we simply need to show
                \begin{align}
                    \sum_{\abs{n} \le N} \abs{a_n e^{2\pi in x_0}} \abs{ e^{2\pi in h} - 1} \le 14MN^2\abs{h}
                \end{align}
                To do this,
                \begin{align}
                    \abs{a_n e^{2\pi in x_0}} &\le \abs{a_n}\\
                    &\le M & \text{Upper bound; $\abs{a_n}$ is abs. cvngt.}
                \end{align}
                and
                \begin{align}
                    \abs{ e^{2\pi in h} - 1} &\le 2\pi n \abs{h}\\
                    &\le 7 N \abs{h}
                \end{align}
                because $\abs{h}$ is small, i.e this difference is smaller than the corresponding arc length, given the angle is very small. So, summing $2N$ of these elements,\footnote{$a_0$ would be the $(2N+1)$-th element, but at $n = 0$, $e^{2\pi in h} - 1 = 0$}
                \begin{align}
                    \sum_{\abs{n} \le N} \abs{a_n e^{2\pi in x_0}} \cdot \abs{ e^{2\pi in h} - 1} \le 14MN^2\abs{h}
                \end{align}
            \end{proof}
            So, based on this, if
            \begin{align}
                \abs{h} < \dfrac{\varepsilon}{14MN^2}
            \end{align}
            then
            \begin{align}
                \abs{f(x_0 + h) - f(x_0)} < \varepsilon + 2\varepsilon = 3\varepsilon
            \end{align}
            So, use $\varepsilon = \varepsilon/3$, and the proof is complete.
        \end{proof}
    \end{enumerate}
\end{proof}