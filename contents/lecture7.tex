\section{Lecture 7}
\subsection{Fourier Series for Real Functions}
Suppose $f: [0,1] \to \mathbb{R}$. Fourier analysis of this function says,
\begin{align}
    f(t) = \sum_{n \in \mathbb{Z}} a_n e^{2\pi int} \mid a_n = \left\langle f(t), e^{2\pi int} \right\rangle
\end{align}
Since this is a real function, we can establish
\begin{lemma}
    For real function $f: [0,1] \to \mathbb{R}$
    \begin{align}
        \overline{a_n} = a_{-n} \mid a_n = \left\langle f(t), e^{2\pi int} \right\rangle
    \end{align}
\end{lemma}
\begin{proof}
    \begin{align*}
        a_n &= \int_0^1 f(t) \overline{e^{2\pi int}} \dd{t}\\
        \overline{a_n} &= \overline{\int_0^1 f(t) \overline{e^{2\pi int}} \dd{t}} = \int_0^1 \overline{f(t) \overline{e^{2\pi int}}} \dd{t}\\
        &= \int_0^1 \overline{f(t)} e^{2\pi int} \dd{t}\\
        &= \int_0^1 f(t) e^{2\pi int} \dd{t} & f(t) \in \mathbb{R}\\
        &= \int_0^1 f(t) \overline{e^{-2\pi int}} \dd{t} = \left\langle f(t), e^{-2\pi int} \right\rangle = a_{-n}
    \end{align*}
\end{proof}
\noindent Using Lemma 7.1,
\begin{align}
    f(t) &= a_0 + \sum_{n > 0} a_n e^{2\pi int} + a_n e^{-2\pi int}\\
    &= a_0 + \sum_{n > 0} a_n e^{2\pi int} + a_n \overline{e^{2\pi int}}
\end{align}
Then, using the radial complex form
\begin{align}
    a_n = r_n \cdot e^{i\theta_n}
\end{align}
where $r_n$ is the radius of the complex vector and $\theta_n$ is its angle relative to $\langle 1, 0\rangle$,
\begin{align}
    f(t) &= a_0 + \sum r_n \left[ e^{2\pi int + i\theta_n} + e^{2\pi int - i\theta_n} \right] & \text{second term uses $-n$}
\end{align}
We are adding a complex number and its conjugate\footnote{$(a + bi) + (a - bi) = 2a$}, so
\begin{align}
    \boxed{f(n) = a_0 + \sum_{n=1}^\infty 2r_n \cos\left[ 2\pi nt + \theta_n \right]}
\end{align}
proving that all real functions can be composed as the sum of trigonometric functions.

\subsection{Digression on Sound}
A sound is some function
\begin{align}
    f(n) = \sum_n a_n e^{2\pi int}
\end{align}
But humans can only hear certain frequencies so the function
\begin{align}
    f(n) = \sum_n^{20000} a_n e^{2\pi int}
\end{align}
sounds the same. We thus only need 40000 numbers $-20000, -19999, ..., 19999, 20000$ to represent a sound at some time $t$ (in order to use 7.7 we need to cancel the exponential).
\subsubsection{Sound Compression}
\texttt{mp3} compression algorithms pick the top $n$ frequencies at a certain time and delete the rest, saving space. On top of this, harmonic analysis can be performed to determine which frequencies are ``cancelled out'' and remove them too, further saving space.
