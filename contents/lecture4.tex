\section{Lecture 4}
\subsection{More on Derivatives and Integrals}
Derivatives represent position, velocity, acceleration, jerk, etc.
\begin{align}
    \dv{}{t} \left(e^{it} \right) = ie^{it}
\end{align}
This is the velocity of the trajectory and multiplying by $i$ induces a $90^\circ$ rotation. Integrals then might represent the center of mass, or more generally, a complete sum which can be used for averages.
\begin{lemma}
    \begin{align}
        \int_0^{2\pi} e^{int}e^{-imt} \dd{t} = \begin{cases}
            0 \implies n\ne m\\
            2\pi \text{ otherwise}
        \end{cases}
    \end{align}
\end{lemma}

\subsection{Inner Product}
The dot product in $\mathbb{R}^n$ is denoted by
\begin{align}
    (x_1, x_2, ..., x_n) \cdot (y_1, y_2, ..., y_n) = \sum_{i=1}^n x_iy_i
\end{align}
For functions, this does not quite work, as there are infinitely many terms; instead, an integral is a better tool to represent the inner product.
\begin{definition}
    The standard inner product on $C_\mathbb{R}[0,1]$ is the function that takes two vectors $f, g \in C_\mathbb{R}[0,1]$ and outputs the number
    \begin{align}
        \langle f, g \rangle := \int_0^1 f(x)g(x) \dd{x}
    \end{align}
\end{definition}
This is the ``infinite dimensional version of the dot product'' and satisfies many of the same properties:
\subsubsection{Properties of a (Real) Inner Product}
\begin{definition}
    Any real inner product satisfies:
    \begin{enumerate}
        \item $\langle f, g \rangle = \langle g, f \rangle$
        \item $\langle \lambda f, g \rangle = \lambda \langle f, g \rangle$
        \item $\langle f + g, h \rangle = \langle f, h \rangle + \langle g, h \rangle$
        \item $\langle f, f \rangle \ge 0$
        \begin{enumerate}
            \item $\langle f, f \rangle = 0 \iff f = 0$
        \end{enumerate}
    \end{enumerate}
\end{definition}
\textbf{Any function that satisfies these four properties is an inner product.} With these we can also talk about length and angle.
\begin{definition}
    If $f, g \in V$\footnote{a vector space}, define the norm of $f$ as
    \begin{align}
        \norm{f} := \sqrt{\langle f, f \rangle}
     \end{align}
     Then, the distance between $f$ and $g$ is $\norm{f-g}$ and the angle between them is the number $\gamma$ such that
     \begin{align}
         \langle f, g \rangle = \norm{f} \cdot \norm{g} \cdot \cos(\gamma)
     \end{align}
     Finally, if the angle between $f$ and $g$ is $\pi/2$ radians, then we say that $f \perp g$ (the inner product must be zero).
\end{definition}