\section{Lecture 23}
\subsection{The Heat Equation, Revisited}
Recall
\begin{align}
    \pdv{u}{t} = \laplacian u
\end{align}
In one dimension, this is
\begin{align}
    \pdv{u}{t} = \pdv[2]{u}{x}
\end{align}
Let $v(t, \xi)$ be the Fourier transform of some solution $u(t, x)$ in the x-direection. Then, taking the FT of both sides,
\begin{align}
    (23.2) \implies \pdv{v(t, \xi)}{t} &= (-i\xi)^2 v(t, \xi)\\
    &= -\xi^2 v(t, \xi)
\end{align}
This makes the PDE into an ODE, specifically
\begin{align}
    \dv{v}{t} = -\xi^2 v
\end{align}
which has the solution
\begin{align}
    v(t, \xi) = e^{-\xi^2t} \cdot v(0, \xi) = e^{-\xi^2 t} \cdot \widehat{h(x)}(\xi)
\end{align}
To rewrite this, recall
\begin{align}
    \mathcal{F}\left[ e^{-x^2/2} \right] = e^{-\xi^2/2}
\end{align}
Dividing the function by a factor of $a$ yields, by previously proven properties
\begin{align}
    \mathcal{F}\left[ e^{-a^2x^2/2} \right] = \dfrac{1}{a} e^{-\xi^2/2a^2}
\end{align}
Since we want this to equal $e^{-\xi^2 t}$, we can use
\begin{align}
    a = \dfrac{1}{\sqrt{2t}}
\end{align}
which would yield
\begin{align}
    \mathcal{F}\left[ e^{-x^2/4t} \right] = \sqrt{2t}e^{-\lambda^2t}
\end{align}
So, (23.6) becomes
\begin{align}
    (23.6) &= \dfrac{1}{\sqrt{2t}} \widehat{e^{-x^2}/4t} \cdot \widehat{h(x)}\\
    &= \dfrac{1}{2\sqrt{\pi t}} \widehat{e^{-x^2/4t} * h(x)} & \text{prev. properties}
\end{align}
Thus, taking $\mathcal{F}^{-1}$,
\begin{align}
    u(t, x) = \underbracket{\left[ \dfrac{1}{2\sqrt{\pi t}} e^{-x^2/4t} \right]}_{\kappa_t\text{ Heat Kernel}} * h(x)
\end{align}
So, the temperature at some point at some time is some sort of weighted average of initial temperatures.
\begin{enumerate}
    \item $t \to t_0$ means $\kappa_t$ ``goes to zero'' very fast, which means the weights are only significant around $t_0$.
    \item $t \to \infty$ means $\kappa_t$ is effectively constant, making the temperature into the average temperature at all points.
\end{enumerate}
But, this model would imply that at some $t = 10^{-50} \text{ s}$, $\Delta t \ne 0$ which is physically impossible due to speed constraints. So, this is still an oversimplification of reality.

\subsection{Multidimensional Functions}
Multidimensional functions are an upgrade from the status quo of living along a single dimension. Using 2D as an example, a wave might look like
\begin{align}
    f(x, y) = e^{ix}
\end{align}
which is a sine wave alone the $[1,0]$ unit vector. In general,
\begin{align}
    f_{[a,b]} = e^{i(ax+by)}
\end{align}
We claim:
\begin{theorem}
    Every reasonable function can be some infinite sum of these waves described in (23.15).
\end{theorem}
To prove this, define \textbf{The Multidimensional Fourier Transform}:
\begin{definition}
    (The Multidimensional Fourier Transform) Let $f: \mathbb{R}^2 \to \mathbb{C}$ be some function. Then, the 2D Fourier transform of $f$ is
    \begin{align}
        \widehat{f}(\xi, \eta) = \dfrac{1}{2\pi} \iint_{\mathbb{R}^2} f(x, y) e^{-i(\xi x + \eta y)} \dd{x}\dd{y}
    \end{align}
    So, in general,
    \begin{align}
        \widehat{f}(a_1, \cdots, a_n) = \dfrac{1}{(2\pi)^{n/2}} \int_{\mathbb{R}^n} f(x_1, \cdots, x_n) e^{-i(\sum_i a_i x_i)} \dd{x_1}\cdots\dd{x_n}
    \end{align}
\end{definition}

\subsection{Properties of a Multidimensional Fourier Transform}
In multiple dimensions, properties still hold
\begin{enumerate}
    \item $\delta_x \widehat{f(x,y)} = -i\xi \hat{f}(\xi, \eta)$
    \item $\widehat{f*g} \propto \hat{f} \cdot \hat{g}$
\end{enumerate}
Wherever there is a $\sqrt{2\pi}$, it now must be taken to the $n$-th power where $n$ is the dimension of the field being transformed. Finally,
\begin{align}
    \norm{f} = \norm{\hat{f}}
\end{align}
also still holds.