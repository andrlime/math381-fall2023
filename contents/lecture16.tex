\section{Lecture 16}
``It's convoluted''
\subsection{Redefining Convolution}
Recall that for periodic functions
\begin{definition}
    The convolution of \textbf{two periodic functions} $f$ and $g$ is
    \begin{align}
        (f * g)(x) := \int_0^1 f(t)\cdot g(x-t) \dd{t}
    \end{align}
\end{definition}
It is quite trivial to prove
\begin{lemma}
    \begin{align}
        a_n(f * g) = a_n(g * f)
    \end{align}
    and
    \begin{align}
        f*g = g*f
    \end{align}
\end{lemma}
via simple commutativity. Note that (16.1) does not apply to non-periodic functions, which would be some integral from $-\infty$ to $+\infty$.
\begin{lemma}
    $$f*g = g*f$$
    but in a different way.
\end{lemma}
\begin{proof}
    \begin{align}
        \int_0^1 f(t)\cdot g(x-t) \dd{t} &= -\int_x^{x-1} f(x-s)\cdot g(s) \dd{s}\\
        &= \int_0^1 f(x-s)\cdot g(s) \dd{s} & \text{It's periodic}
    \end{align}
\end{proof}
\noindent Blurring is an example of a convolution.
\begin{align}
    (B_\varepsilon f)(x) = \frac{1}{\varepsilon} \int_0^\varepsilon f(x - s) \dd{s}
\end{align}
We can define
\begin{align}
    1_{[0,\varepsilon]}(x) := \begin{cases}
        1 & x \in [0, \varepsilon]\\
        0 & \text{else}
    \end{cases}
\end{align}
And it can be shown that
\begin{align}
    (B_\varepsilon f)(x) = \left(f * \left[ \frac{1}{\varepsilon} 1_{[0,\varepsilon]} \right]\right)
\end{align}

\subsection{A Black Box}
Suppose there is a signal
\begin{align}
    f: \mathbb{R} \to \mathbb{C}
\end{align}
a system
\begin{align}
    T
\end{align}
and a response
\begin{align}
    Tf: \mathbb{R} \to \mathbb{C}
\end{align}
from the signal.
\subsection{Time Variance}
Suppose there is some system
\begin{align}
    S_a
\end{align}
that time shifts an input function
\begin{align}
    f(x)
\end{align}
into
\begin{align}
    f(x + a)
\end{align}
\begin{definition}
    A system $T$ is called
    \begin{enumerate}
        \item \textbf{Time Invariant} if $TS_a = S_aT$.
        \item \textbf{Linear} if $T(\lambda f+g) = \lambda T(f) + T(g)$, which holds true in approximations, but is generally not true.
    \end{enumerate}
\end{definition}
\subsection{The Fundamental Theorem of Engineering}
\begin{theorem}
    \textbf{(The Fundamental Theorem of Engineering)} Suppose $T$ is linear and time-invariant. Then,
    \begin{enumerate}
        \item $T \left( e^{2\pi in t} \right)$ is always a multiple of $e^{2\pi in t}$, i.e.
        \begin{align}
            e^{2\pi in t}
        \end{align}
        are the eigenvalues of $T$.
        \item If
        \begin{align}
            \sum_{n \in \mathbb{Z}} \abs{\lambda_n} < \infty
        \end{align}
        then there exists a function $g$ such that
        \begin{align}
            Tf = f * g
        \end{align}
        i.e. all linear, time-invariant systems are just convolutions.
    \end{enumerate}
\end{theorem}
\subsection{The Proof}
First, the eigenvalues.
\begin{proof}
    Denote $f_n = T e^{2\pi in t}$. Then,
    \begin{align}
        S_a T e^{2\pi in t} = T S_a e^{2\pi in t}
    \end{align}
    which becomes
    \begin{align}
        S_a f_n &= T\left[ e^{2\pi in (t+a)} \right]\\
        &= T\left[ e^{2\pi in t} \cdot e^{2\pi in a} \right]\\
        &= e^{2\pi in a} \cdot T\left[ e^{2\pi in t} \right]\\
        &= e^{2\pi in a} \cdot f_n
    \end{align}
    At $t=0$,
    \begin{align}
        f_n(0 + a) &= (S_a f_n)(0)\\
        &= e^{2\pi in a} \cdot f_n(0)
    \end{align}
    Hence, let\footnote{Some variable is wrong here.} $f_n(0) := \lambda_n$
    \begin{align}
        f_n(a) = \lambda_n \cdot e^{2\pi in a} \implies f_n = \lambda_n \cdot e^{2\pi in t}
    \end{align}
    so
    \begin{align}
        T(e^{2\pi in t}) = \lambda_n \cdot e^{2\pi in t}
    \end{align}
    proving that the result of applying the linear transformation is a multiple of that function.
\end{proof}
\noindent Next, that $T$ is a convolution.
\begin{proof}
    Let $g(t) = \sum \lambda_n \cdot e^{2\pi in t}$. We claim
    \begin{align}
        Tf = f*g \mid \forall f
    \end{align}
    By definition, $g(t)$ is continuous. If we can show
    \begin{align}
        a_n(Tf) = a_n(f * g)
    \end{align}
    then we will have proven that the two are the same.
    \begin{align}
        f(t) &= \sum a_n(f) \cdot e^{2\pi in t}\\
        Tf &= T\left[ \sum a_n(f) \cdot e^{2\pi in t} \right]\\
        &= \sum a_n(f) \cdot T(e^{2\pi in t}) & \text{linearity}\\
        &= \sum a_n(f) \cdot \lambda_n \cdot e^{2\pi in t} & \text{Part 1 of Theorem 17.5}
    \end{align}
    So,
    \begin{align}
        a_n(Tf) = a_n(f) \cdot \lambda_n
    \end{align}
    but $\lambda_n$ are just the Fourier coefficients of $g(t)$ by definition, so
    \begin{align}
        a_n(Tf) = a_n(f) \cdot a_n(g) = a_n(f*g) && \text{PSet 4 Problem 2}
    \end{align}
    So,
    \begin{align}
        Tf = f * g && \text{Fourier coefficients equal}
    \end{align}
\end{proof}