\section{Lecture 1}
We can define STEM as subjects where we use Fourier Analysis.
\subsection{Some Motivation}
We can define the trajectory of a planet as the sum of $n$ cyclic motions
\begin{align}
    \vb{\gamma}(t) = \underset{\text{cycle}}{\vb{\gamma_1}(t)} + \underset{\text{epicycle}}{\vb{\gamma_2}(t)} + ... + \vb{\gamma_n}(t)
\end{align}
All functions can be approximated by the sum of many cyclic motions, which is Fourier analysis. In essence, Fourier analysis gives us a new way of thinking about functions.

\subsection{1D Example}
Suppose we have a linear cyclic motion; how can we make this with cycles? The sum of two vector functions
\begin{align}
    \vb{f}(t) &:= \langle \cos(t), \sin(t) \rangle\\
    \vb{g}(t) &:= \langle \cos(-t), \sin(-t) \rangle = \langle \cos(t), -\sin(t) \rangle \\
    \vb{h}(t) &:= \vb{f}(t) + \vb{g}(t) = \langle 2\cos(t), 0 \rangle
\end{align}

\subsection{Sets}
\begin{definition}
    A set is an unordered collection. There are many ways to specify a set:
    \begin{enumerate}
        \item As a group of a few elements:
        \begin{enumerate}
            \item $\{ 1, 2, 3 \}$
            \item $\{ 1, 2, 3, 4, 5 \}$
        \end{enumerate}
        \item As a group of a lot of elements:
        \begin{enumerate}
            \item $\{ 1, 2, 3, ..., 1000 \}$
            \item $\{ 1, 2, 3, ... \}$
        \end{enumerate}
        \item Using set builder notation:
        \begin{enumerate}
            \item $\{ x : \text{some condition for $x$} \}$
            \item $\{ n : n \text{ is the square of some integer $m$} \}$
        \end{enumerate}
    \end{enumerate}
\end{definition}
Sets are unordered, meaning
\begin{align}
    \{ 1, 2 \} = \{ 2, 1 \}
\end{align}
contrasting with sequences which are ordered, meaning
\begin{align}
    (1, 2) \ne (2, 1)
\end{align}
\subsubsection{Some common sets}
There are many common sets, such as
\begin{enumerate}
    \item $\mathbb{N}$ for natural numbers $\{ 0, 1, 2, 3, ... \}$
    \item $\mathbb{Z}$ for integers $\{ ..., -2, -1, 0, 1, 2, ... \}$
    \item $\mathbb{R}$ for real numbers
    \item $\mathbb{C}$ for complex numbers
\end{enumerate}
An element $n$ is said to be in a set $\mathbb{A}$, i.e. $n \in \mathbb{A}$, if $n$ is an element of $\mathbb{A}$. Note that $\in$ is not necessarily a statement of fact, so $\frac{1}{2}\in \mathbb{Z}$ is a perfectly valid statement; it's just false.

\subsection{Functions}
A function maps one set to another, not necessarily different set. For example,
\begin{align}
    f: \mathbb{A} \to \mathbb{B}
\end{align}
is a function mapping an input $x \in \mathbb{A}$ to an output $y \in \mathbb{B}$.
\subsection{Tuples}
A tuple, for example
\begin{align}
    (a_1, a_2, ..., a_n) && \text{n-tuple}
\end{align}
is actually a special kind of function
\begin{align}
    a: \{ 1, 2, ..., n \} \to \mathbb{A}
\end{align}
Each number in $\{ 1, 2, ..., n \}$ corresponds to a specific $a_i \in \mathbb{A}$.

\subsection{Complex Numbers}
The set of complex numbers is the set $\mathbb{C}$.
\begin{definition}
    Complex numbers are expressions of the form
    \begin{align}
        a + bi
    \end{align}
    where $a, b \in \mathbb{R}$ and $i := \sqrt{-1}$
\end{definition}
So, $\mathbb{C}$ can be expressed in set builder notation as
\begin{align}
    \mathbb{C} := \{ a+bi : a,b \in \mathbb{R} \}
\end{align}
From this there are two functions
\begin{align}
    \text{Re}: \mathbb{C} \to \mathbb{R} \Rightarrow \text{Re}(a+bi):=a\\
    \text{Im}: \mathbb{C} \to \mathbb{R} \Rightarrow \text{Im}(a+bi):=b
\end{align}
to take the real and imaginary components from a complex number. Further, define the conjugate of a complex number as
\begin{align}
    \overline{(a+bi)} := (a-bi)
\end{align}
Then,
\begin{align}
    (a+bi)\overline{(a+bi)} = a^2 + b^2
\end{align}
At a more fundamental level, this is true because $\mathbb{C}$ is a vector space, in which addition is defined through
\begin{align}
    (a+bi) + (c+di) := (a+c) + (b+d)i
\end{align}
and multiplication is defined through distribution:
\begin{align}
    (a+bi)(c+di) := ac + (ad + bc)i - bd
\end{align}
Finally, the absolute value of a complex number is defined as
\begin{align}
    |a + bi| := \sqrt{(a+bi)\overline{(a+bi)}}
\end{align}
Indeed, when $b=0$, $|a| = \sqrt{a^2}$, which is the same as for $\mathbb{R}$.