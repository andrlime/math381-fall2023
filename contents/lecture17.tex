\section{Lecture 17}
``One of our main tools to show convergence is to strength it and show absolute convergence.''
\subsection{The Fourier Transform}
\begin{definition}
    Let $f: \mathbb{R}\to\mathbb{C}$. The, the Fourier Transform of $f$ is given by
    \begin{align}
        \hat{f}(\xi) = \dfrac{1}{\sqrt{2\pi}} \int_{-\infty}^\infty f(x) \cdot e^{-ix\xi} \dd{x}
    \end{align}
\end{definition}
\begin{definition}
    Let $V$ be the collection of functions
    \begin{align}
        f: \mathbb{R}\to\mathbb{C}
    \end{align}
    that satisfy
    \begin{enumerate}
        \item There is $M$ such that $\abs{f(x)} \le M$
        \item $f$ is piecewise differentiable
        \item $\int_{-\infty}^\infty \abs{f(x)} \dd{x} < \infty$
    \end{enumerate}
\end{definition}

\subsection{Several Lemmas}
\begin{lemma}
    $V$ is a vector space.
\end{lemma}
\begin{proof}
    The first two are trivial\footnote{Nir's words not mine}. The only non obvious claim is that if
    \begin{align}
        \int_{-\infty}^\infty \abs{f(x)} \dd{x} < \infty \qand \int_{-\infty}^\infty \abs{g(x)} \dd{x} < \infty
    \end{align}
    then
    \begin{align}
        \int_{-\infty}^\infty \abs{f(x) + g(x)} \dd{x} < \infty
    \end{align}
    By the Triangle Inequality,
    \begin{align}
        \int_{-\infty}^\infty \abs{f(x) + g(x)} \dd{x} &\le \int_{-\infty}^\infty \abs{f(x)} + \abs{g(x)} \dd{x}\\
        &\le \int_{-\infty}^\infty \abs{f(x)} \dd{x} + \int_{-\infty}^\infty \abs{g(x)} \dd{x}
    \end{align}
    Both of these are bounded, so
    \begin{align}
        \int_{-\infty}^\infty \abs{f(x) + g(x)} \dd{x} < \infty
    \end{align}
\end{proof}
\begin{lemma}
    If $f, g \in V$, then
    \begin{enumerate}
        \item $\int_{-\infty}^\infty \abs{f(x)}^2 \dd{x} < \infty$
        \item $\int_{-\infty}^\infty f(x)\cdot g(x) \dd{x} < \infty$
    \end{enumerate}
\end{lemma}
\begin{proof}
    First, to show
    \begin{align}
        \int_{-\infty}^\infty \abs{f(x)}^2 \dd{x} < \infty
    \end{align}
    let $f \in V$, then
    \begin{align}
        \abs{f(x)} \le M
    \end{align}
    Then,
    \begin{align}
        \int_{-\infty}^\infty \abs{f(x)}^2 \dd{x} &\le \int_{-\infty}^\infty M \cdot \abs{f(x)} \dd{x}\\
        &\le M\cdot\int_{-\infty}^\infty \abs{f(x)} \dd{x}\\
        &\le \infty & \int_{-\infty}^\infty \abs{f(x)} \dd{x} < \infty
    \end{align}
    Second, to show
    \begin{align}
        \int_{-\infty}^\infty f(x)\cdot g(x) \dd{x} < \infty
    \end{align}
    we can show that
    \begin{align}
        \int_{-\infty}^\infty \abs{f(x)\cdot g(x)} \dd{x} < \infty
    \end{align}
    i.e. show absolute convergence. Then,
    \begin{align}
        \int_{-\infty}^\infty \abs{f(x)\cdot g(x)} \dd{x} &\le \int_{-\infty}^\infty \abs{f(x)} \cdot \abs{g(x)} \dd{x}\\
        &\le M \cdot \int_{-\infty}^\infty \abs{g(x)} \dd{x} < \infty
    \end{align}
\end{proof}
\begin{definition}
    The inner product on $V$ is
    \begin{align}
        \langle f, g \rangle = \int_{-\infty}^\infty f(x) \cdot \overline{g(x)} \dd{x}
    \end{align}
    and converges because of Lemma 18.4.
\end{definition}

\subsection{Some Analogue Proofs}
\begin{lemma}
    If $f \in V$, then $\hat{f}$ exists and $\hat{f}$ is continuous.
\end{lemma}
\begin{proof}
    We \textbf{first want to show $\hat{f}$ exists}, so we can show it is finite
    \begin{align}
        \int_{-\infty}^\infty f(x) \cdot e^{-ix\xi} \dd{x} < \infty
    \end{align}
    We can show absolute convergence
    \begin{align}
        \int_{-\infty}^\infty \abs{f(x) \cdot e^{-ix\xi}} \dd{x} < \infty
    \end{align}
    Indeed,
    \begin{align}
        \int_{-\infty}^\infty \abs{f(x) \cdot e^{-ix\xi}} \dd{x} &\le \int_{-\infty}^\infty \abs{f(x)} \cdot \abs{e^{-ix\xi}} \dd{x}\\
        &= \int_{-\infty}^\infty \abs{f(x)} \dd{x} < \infty
    \end{align}
    by $f \in V$, so (17.18) absolutely converges and thus also converges. \textbf{Next, to show continuity}, we want to show
    \begin{align}
        \lim_{\xi \to \eta} \hat{f}(\xi) = \hat{f}(\eta)
    \end{align}
    or
    \begin{align}
        \lim_{\xi \to \eta} \dfrac{1}{\sqrt{2\pi}} \int_{-\infty}^\infty f(x) \cdot e^{ix\xi} \dd{x} = \dfrac{1}{\sqrt{2\pi}} \int_{-\infty}^\infty f(x) \cdot e^{ix\eta} \dd{x}
    \end{align}
    We can prove this using $\varepsilon-\delta$; or, if $\abs{\xi - \eta}$ is very small, then
    \begin{align}
        \abs{\dfrac{1}{\sqrt{2\pi}} \int_{-\infty}^\infty f(x) \cdot \left[e^{ix\xi} - e^{ix\eta}\right] \dd{x}} < \varepsilon
    \end{align}
    \begin{proof}
        Fix $\varepsilon$. Then, by assumption,
        \begin{align}
            \int_{-\infty}^\infty \abs{f(x)} \dd{x} < \infty
        \end{align}
        This means that there is some $T$ such that
        \begin{align}
            \int_{-\infty}^\infty \abs{f(x)} \dd{x} \approx \int_{-T}^T \abs{f(x)} \dd{x}
        \end{align}
        which would imply
        \begin{align}
            \left[\int_{-\infty}^{-T} \abs{f(x)} \dd{x} \qand \int_{T}^\infty \abs{f(x)} \dd{x}\right] < \varepsilon
        \end{align}
        Next,
        \begin{align}
            \abs{\hat{f}(\xi) - \hat{f}(\eta)} = \abs{\int_{-\infty}^\infty f(x) \left[ e^{ix\xi} - e^{ix\eta} \right] \dd{x}}
        \end{align}
        Now comes the trick of the proof. We rewrite the integral in (17.28) as
        \begin{align}
            \int_{-\infty}^{-T} \cdots + \int_{-T}^{T} \cdots + \int_{T}^{\infty} \cdots
        \end{align}
        Define $\zeta := \left[ e^{ix\xi} - e^{ix\eta} \right]$. Then, using the Triangle Inequality,
        \begin{align}
            (17.28) &\le \int_{-\infty}^{-T} \abs{f(x)} \cdot \abs{\zeta} \dd{x} + \int_{-T}^{T} \abs{f(x)} \cdot \abs{\zeta} \dd{x} + \int_{T}^{\infty} \abs{f(x)} \cdot \abs{\zeta} \dd{x}
        \end{align}
        But, $\abs{\zeta} \le 2$. So,
        \begin{align}
            (17.28) &\le 2\int_{-\infty}^{-T} \abs{f(x)} \dd{x} + \int_{-T}^{T} \abs{f(x)} \cdot \abs{\zeta} \dd{x} + 2\int_{T}^{\infty} \abs{f(x)} \dd{x}
        \end{align}
        Using (17.27),
        \begin{align}
            (17.28) &\le 4\varepsilon + \int_{-T}^{T} \abs{f(x)} \cdot \abs{\zeta} \dd{x}\\
            &\le 4\varepsilon + \int_{-T}^{T} \abs{f(x)} \cdot \abs{e^{ix\xi} - e^{ix\eta}} \dd{x}\\
            &\le 4\varepsilon + \int_{-T}^{T} M \cdot \abs{x} \cdot \abs{\xi - \eta} \dd{x}\\
            &\le 4\varepsilon + 2TM \cdot T \cdot \abs{\xi - \eta}\\
            &\le 4\varepsilon + 2T^2M \cdot \abs{\xi - \eta}
        \end{align}
        So
        \begin{align}
            \abs{\xi - \eta} < \frac{\varepsilon}{2T^2M} \implies \abs{\hat{f}(\xi) - \hat{f}(\eta)} < 5\varepsilon
        \end{align}
        so if $\varepsilon = \frac{\varepsilon}{5}$, then
        \begin{align}
            \abs{\hat{f}(\xi) - \hat{f}(\eta)} < \varepsilon
        \end{align}
    \end{proof}
\end{proof}
