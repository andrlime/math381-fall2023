\section{Lecture 22}
Use the function
\begin{align}
    \sqrt{2\pi} \cdot \widehat{e^{-\abs{x}}}(\xi) &= \int_{-\infty}^\infty e^{-\abs{x}} e^{-ix\xi} \dd{x}\\
    &= \int_{-\infty}^0 e^{x} e^{-ix\xi} \dd{x} + \int_{0}^\infty e^{-x} e^{-ix\xi} \dd{x}\\
    &= \int_{-\infty}^0 e^{x(1-i\xi)} \dd{x} + \int_{0}^\infty e^{x(-1-i\xi)} \dd{x}\\
    &= \eval(\dfrac{e^{x(1-i\xi)}}{1-i\xi}|_{-\infty}^0 + \eval(\dfrac{e^{x(-1-i\xi)}}{-1-i\xi}|_0^\infty = \boxed{\dfrac{1}{1-i\xi} + \dfrac{1}{1+i\xi}}
\end{align}
Simplify that result to get
\begin{align}
    \dfrac{1}{1-i\xi} + \dfrac{1}{1+i\xi} &= \boxed{\dfrac{2}{1+\xi^2}}
\end{align}
So,
\begin{align}
    \widehat{e^{-\abs{x}}}(\xi) = \sqrt{\dfrac{2}{\pi}} \dfrac{1}{1+\xi^2}
\end{align}

\subsection{Going Back to Differential Equations}
Recall
\begin{align}
    \widehat{f'(x)}(\xi) = -i\xi \hat{f}(\xi)
\end{align}
Conversely, take the derivative of a Fourier transform
\begin{align}
    \dv{\xi} \hat{f}(\xi) &= \dfrac{1}{\sqrt{2\pi}} \dv{\xi} \int f(x) e^{-ix\xi} \dd{x}\\
    &= \dfrac{1}{\sqrt{2\pi}} \cdot \int f(x) \cdot (-ix) e^{-ix\xi} \dd{x}\\
    &= \widehat{-ix f(x)}(\xi)
\end{align}
Use this in solving differential equations. Suppose
\begin{align}
    -f''(x) + f(x) = h(x)
\end{align}
Take the Fourier transform of both sides
\begin{align}
    -(-i\xi)^2 \hat{f}(\xi) + \hat{f}(\xi) = \hat{h}(\xi)
\end{align}
so
\begin{align}
    (1 + \xi^2) \hat{f}(\xi) = \hat{h}(\xi)
\end{align}
the solution to this is
\begin{align}
    \hat{f}(\xi) = \dfrac{1}{1+\xi^2} \cdot \hat{h}(\xi)
\end{align}
But, use (22.6). This becomes
\begin{align}
    \hat{f}(\xi) &= \widehat{\sqrt{\frac{\pi}{2}} e^{-\abs{x}}} \cdot \hat{h}(\xi) \cdot \sqrt{2\pi} \cdot \dfrac{1}{\sqrt{2\pi}}\\
    &= \dfrac{1}{2} \left[ \sqrt{2\pi} \hat{h}(\xi) \cdot \widehat{e^{-\abs{x}}} \right]\\
    &= \dfrac{1}{2} \left[ \widehat{h(x) * e^{-\abs{x}}}(\xi) \right]
\end{align}
So,
\begin{align}
    f(x) &= \dfrac{1}{2} \left[ h(x) * e^{-\abs{x}} \right]\\
    &= \dfrac{1}{2} \int_{-\infty}^\infty h(t) \cdot e^{-\abs{x-t}} \dd{t}
\end{align}

\subsection{Airy Equation}
\begin{align}
    f''(x) - xf(x) = 0
\end{align}
Take the Fourier transform and get
\begin{align}
    (-i\xi)^2 \hat{f}(\xi) - \dfrac{1}{i}\dv{\xi} \hat{f}(\xi) = 0
\end{align}
which becomes
\begin{align}
    \dfrac{1}{i}\dv{\xi} \hat{f}(\xi) = -\xi^2 \hat{f}(\xi)
\end{align}
or, if $g = \hat{f}$,
\begin{align}
    g'(\xi) = i\xi^2 g(\xi)
\end{align}
Then, rearranging, we get
\begin{align}
    \dv{\xi} \log{\hat{f}(\xi)} = -i\xi^2 \implies \log{\hat{f}(\xi)} = -\dfrac{i\xi^3}{3} + C
\end{align}
This implies
\begin{align}
    \hat{f}(\xi) = D \cdot \exp\left[ -\frac{i\xi^3}{3} \right]
\end{align}
So, taking the inverse FT,
\begin{align}
    f(x) &= \dfrac{D}{\sqrt{2\pi}} \int_{-\infty}^\infty e^{-\frac{i\xi^3}{3} + ix\xi} \dd{\xi}\\
    &= F \int_{-\infty}^\infty e^{-i(\frac{\xi^3}{3} + x\xi)} \dd{\xi}
\end{align}

\subsection{A Third Example}
Start with the function
\begin{align}
    f(x) = e^{-x^2/2}
\end{align}
This function satisfies a differential equation
\begin{align}
    f'(x) + xf(x) = 0
\end{align}
Conversely,
\begin{align}
    g \text{ satisfies (22.29)} \implies g = Cf
\end{align}
Take the FT of (22.29) and get
\begin{align}
    (-i\xi) \hat{f}(\xi) - i \dv{\xi} \hat{f}(\xi) = 0
\end{align}
Rearranging, $\hat{f}$ also satisfies the differential equation in (22.29) which implies
\begin{align}
    \hat{f} = Cf
\end{align}
What is this constant? By Bessel\footnote{man},
\begin{align}
    \norm{e^{-x^2/2}} = \norm{C \cdot e^{-\xi^2/2}} = \abs{C} \cdot \norm{e^{-\xi^2/2}} \implies C = \pm1
\end{align}
But $C = 1$ because the integrand is positive. This also implies
\begin{align}
    \int_{-\infty}^\infty e^{-x^2/2} \dd{x} = \sqrt{2\pi}
\end{align}