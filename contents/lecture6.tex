\section{Lecture 6}
\subsection{Convergence}
Instead of convergence of points, convergence of functions, which is the reason infinite dimensional vector spaces are important.
\subsection{Tangent: Triangle Inequality and Cauchy-Schwartz}
\begin{definition}
If $V$ is a vector space with an inner product, and $\vb{u}, \vb{v} \in V$, then
\begin{align}
    \norm{\vb{u} + \vb{v}} \le \norm{\vb{u}} + \norm{\vb{v}}
\end{align}
\end{definition}
\begin{definition}
    Cauchy-Schwartz Inequality\footnote{It's quite odd that this was framed as a definition in lecture, but I will leave it as is.}
    \begin{align}
        \abs{\langle \vb{u}, \vb{v} \rangle} \le \norm{\vb{u}}\norm{\vb{v}}
    \end{align}
    \begin{proof}
        In the real case,
        \begin{align}
            f(t) :=& \norm{\vb{u} + t\vb{v}}^2 = \langle \vb{u} + t\vb{v}, \vb{u} + t\vb{v} \rangle\\
            =& \norm{\vb{u}}^2 + 2t \langle \vb{u}, \vb{v} \rangle + t^2 \norm{\vb{v}}^2 \ge 0\\
            \implies & \text{$\Delta \le 0$} & \text{1 sol. or 0 sol.}\\
            \implies & 4\langle \vb{u}, \vb{v} \rangle ^2 - 4\norm{\vb{u}}^2\norm{\vb{v}}^2 \le 0\\
            \implies & \langle \vb{u}, \vb{v} \rangle ^2 \le \norm{\vb{u}}^2\norm{\vb{v}}^2
        \end{align}
    \end{proof}
\end{definition}

\subsection{Back to Convergence}
\begin{definition}
    Let $V$ be a vector space with an inner product. Let $\vb{v}_1, \vb{v}_2, ...$ be a sequence of vectors. We say
    \begin{align}
        &\lim_{n\to\infty} \vb{v}_n = \vb{v}_\infty \impliedby \lim_{n\to\infty} \norm{\vb{v}_n - \vb{v}_\infty} = 0\\
        &\sum_1^\infty \vb{v}_n = \vb{w} \impliedby \vb{v}_1, \vb{v}_1 + \vb{v}_2, ..., \vb{v}_1 + \vb{v}_2 + ... + \vb{v}_n \to \vb{w}
    \end{align}
\end{definition}
\begin{lemma}
    If $\vb{v}_n \to \vb{v}_\infty$, then $\norm{\vb{v}_n} \to \norm{\vb{v}_\infty}$
\end{lemma}
\begin{proof}
    Apply $\Delta$ inequality to $\vb{v}_\infty$ and $\vb{v}_n - \vb{v}_\infty$. So
    \begin{align}
        \norm{\vb{v}_n} \le \norm{\vb{v}_\infty} + \norm{\vb{v}_n - \vb{v}_\infty}
    \end{align}
    Take limit of both sides and
    \begin{align}
        \lim \norm{\vb{v}_n} \le \norm{\vb{v}_\infty} + \lim \norm{\vb{v}_n - \vb{v}_\infty}
    \end{align}
    We can then apply the $\Delta$ inequality the other way and we get
    \begin{align}
        &\norm{\vb{v}_\infty} \le  \lim \norm{\vb{v}_n} + \lim \norm{\vb{v}_\infty - \vb{v}_n}
    \end{align}
    \begin{align}
&\boxed{            \norm{\vb{v}_\infty} \le \lim \norm{\vb{v}_n} \le \norm{\vb{v}_\infty}
\implies            \lim \norm{\vb{v}_n} = \norm{\vb{v}_\infty}}
    \end{align}
\end{proof}

\begin{definition}
    Suppose $\{ \vb{v}_n \}$ is an orthonormal sequence in $V$. We say that $\{ \vb{v}_n \}$ is closed (or that they are an orthonormal basis) if for every $\vb{v} \in V$,
    \begin{align}
        \vb{v} = \sum_{i=1}^\infty \langle \vb{v}, \vb{v}_i \rangle \vb{v}_i
    \end{align}
\end{definition}

\subsection{Main Theorem of This Course}
\begin{theorem}
    The set
    \begin{align}
        \{ e^{2\pi int} \}
    \end{align}
    forms a closed, orthonormal set of functions in $C[0,1]$.
\end{theorem}
